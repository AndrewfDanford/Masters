\chapter{Problem Setup and Scope}

\section{Research Questions}
\begin{enumerate}
  \item Under a unified protocol, which explanation family (saliency, concept, or text) is most faithful for chest X-ray (CXR) diagnosis models?
  \item Does adding explicit concept grounding improve the faithfulness of textual rationales?
  \item Are cross-family faithfulness conclusions stable under perturbation choice, model backbone choice, and label uncertainty handling?
\end{enumerate}

\section{Hypotheses}
\textbf{H1 (primary).} Under matched predictive performance, explanation methods with explicit concept grounding achieve higher faithfulness than purely post-hoc saliency methods.

\textbf{H2.} Constrained textual rationales are less fluent but more faithful than less-constrained textual rationales.

\textbf{H3.} A unified benchmark protocol reveals method ranking differences that are not visible when each explanation family is evaluated in isolation.

\section{Formal Task Definition}
Let $\mathcal{D}=\{(x_i, y_i, s_i, r_i)\}_{i=1}^{N}$ denote CXR data, where:
\begin{itemize}
  \item $x_i$ is a chest X-ray image,
  \item $y_i \in \{0,1\}^{K}$ is a $K$-label pathology vector,
  \item $s_i$ denotes study metadata (patient, study, and view fields),
  \item $r_i$ is optional report text.
\end{itemize}

The diagnostic model $f_{\theta}$ predicts $\hat{y}_i=f_{\theta}(x_i)$. An explanation generator $g_{\phi}$ produces $e_i=g_{\phi}(x_i, f_{\theta})$, where $e_i$ can be:
\begin{itemize}
  \item a saliency map,
  \item a concept vector,
  \item a short textual rationale.
\end{itemize}

The central evaluation objective is not only predictive quality of $f_{\theta}$ but the behavioral faithfulness of $e_i$ under controlled interventions.

\section{Data Scope and Rationale}
The primary dataset stack is:
\begin{itemize}
  \item \textbf{MIMIC-CXR-JPG} for large-scale CXR images, report text, and metadata \citep{johnson2019mimiccxr}.
  \item \textbf{MS-CXR} for localization-aligned phrase annotations supporting region-level evaluation \citep{boecking2024mscxr}.
  \item \textbf{RadGraph} for report-derived entities and relations used to define concept-grounded explanations \citep{jain2021radgraph}.
\end{itemize}

This stack is selected because it supports all three explanation families within one domain and enables a clinically relevant comparison rather than synthetic cross-dataset mixing.

\section{Cohort and Label Scope}
To limit confounding and preserve feasibility in a two-semester timeline:
\begin{itemize}
  \item Only frontal views (AP/PA) are included.
  \item Splits are patient-level to prevent leakage.
  \item Initial findings are: Atelectasis, Cardiomegaly, Consolidation, Edema, Pleural Effusion, and Pneumothorax.
  \item The default uncertain-label policy is to exclude uncertain labels from prevalence denominators in primary analyses.
\end{itemize}

\section{Threat Model and Validity Risks}
The benchmark explicitly addresses three validity risks:
\begin{itemize}
  \item \textbf{Shortcut risk:} explanation highlights non-clinical correlates (for example, acquisition artifacts).
  \item \textbf{Plausibility risk:} explanation appears convincing but is weakly coupled to model behavior.
  \item \textbf{Metric-fragmentation risk:} conclusions change substantially across incompatible evaluation definitions.
\end{itemize}

The design therefore prioritizes intervention-based metrics, standardized perturbation settings, and paired comparisons across explanation families.

\section{Scope Boundaries}
This thesis is a computer science benchmark study and makes no claim of immediate clinical deployment. It evaluates explanation validity with computational and dataset-based evidence, while treating full clinician reader studies as future work.

\section{Non-Goals}
\begin{itemize}
  \item No claim of bedside deployment readiness.
  \item No claim of causal medical mechanism discovery.
  \item No requirement for a full radiologist reader study in the master's timeline.
  \item No claim that generated rationales are equivalent to formal radiology reports.
\end{itemize}
